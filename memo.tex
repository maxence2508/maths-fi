\documentclass[8pt,twocolumn]{extarticle}
\usepackage{graphicx}
\usepackage[utf8]{inputenc}
\usepackage{amsmath}
\usepackage{amsfonts}
\usepackage[a4paper, margin=0.7cm]{geometry}

% Macros
\newcommand{\E}{\mathbb{E}}
\newcommand{\prob}{\mathbb{P}}
\newcommand{\f}{\mathcal{F}}
\newcommand{\norm}{\mathcal{N}}
\newcommand{\n}{\mathbb{N}}

\renewcommand{\labelitemi}{--}
\setlength{\parindent}{0pt}

\title{\Huge Modèle de Black--Scholes}
\author{Maxence Caucheteux}
\date{}

\begin{document}

\maketitle

\section{Résultats sur le modèle}
\subsection*{Présentation du modèle}
Ce modèle comprend \underline{un actif sans risque} $S^0_t = e^{rt}$ et \underline{un actif risqué} $S_t$.
La dynamique de l'actif risqué est régie par l'EDS :
$$dS_t = S_t (\mu dt + \sigma d B_t), \ \ S_0 = s_0$$
avec $\mu \in \mathbb{R}, \sigma > 0$.

\subsection*{Résolution de l'EDS}
L'EDS admet une unique solution forte par théorème d'Itô et c'est :
$$S_t = s_0 \exp \left( (\mu - \sigma^2/2)t + \sigma dB_t \right)$$

\subsection*{Portefeuille ou stratégie}
C'est un processus adapté $\phi = (H^0_t, H_t)_{t \in [0,T]}$. Sa valeur est $V_t (\phi) = H^0_t e^{rt} + H_t S_t$. 

Portefeuille autofinancé si :
\begin{itemize}
    \item $\int_0^T (|H^0_t| + H^2_t) dt < + \infty \ \ \prob$-ps
    \item $V_t (\phi) = V_0 + \int_0^t H^0_u dS^0_u + \int_0^t H_t dS_t$, équivalent à $d\tilde{V}_t(\phi) = H_t d\tilde{S}_t$
\end{itemize}

Stratégie admissible si :
\begin{itemize}
    \item autofinancée
    \item $\tilde{V}_t(\phi) \geq 0$
    \item $\E^* \left( \sup_{t \in [0,t]} \tilde{V}_t^2 \right) < + \infty$
\end{itemize}

\subsection*{Changement de probabilité}
On pose, pour $t \in [0,T]$ :
$$W_t = B_t + \frac{\mu-r}{\sigma}t, \ \ L_t = \exp \left(- \frac{\mu-r}{\sigma} B_t - \frac{1}{2} \left(\frac{\mu-r}{\sigma} \right)^2t \right)$$
Par théorème de Girsanov appliqué à $\theta_s = (\mu-r)/\sigma$, sous $\prob^*$ de densité $\frac{d\prob^*}{d\prob} = L_T$, $(W_t)$ est un $(\f_t)$-mouvement brownien.

\subsection*{Viabilité du marché}
Le marché est viable dans le modèle de Black-Scholes i.e. pour tout $h \geq 0$ $\f_T$-mesurable de carré intégrable, il existe $\phi$ admissible simulant cette option. De plus, pour tout option $\phi$ simulant l'option $h$, on a :
$$V_t = \E^* \left( e^{-r(T-t)} h \ \middle| \ \f_t \right)$$

\subsection*{Pricing}
\begin{table}[h!]
\centering
\resizebox{\columnwidth}{!}{%
\begin{tabular}{|c||c|c|}
\hline
 & Call & Put \\
\hline\hline
Prix $F(t,x)$
& $x\,\norm\!\bigl(d_1(\theta,x)\bigr)
   - K e^{-r\theta}\,\norm\!\bigl(d_2(\theta,x)\bigr)$
& $K e^{-r\theta}\,\norm\!\bigl(-d_2(\theta,x)\bigr)
   - x\,\norm\!\bigl(-d_1(\theta,x)\bigr)$ \\
\hline
Delta $\partial_x F(t,x)$
& $\norm\!\bigl(d_1(\theta,x)\bigr)$
& $-\norm\!\bigl(-d_1(\theta,x)\bigr)$ \\
\hline
\end{tabular}
}
\caption{Prix et delta}
\end{table}

Avec $\theta = T-t$ et :
$$d_1 (\theta, x) = \frac{1}{\sigma \sqrt\theta} \left( \log \left( \frac{x}{K} \right) + \left( r + \frac{\sigma^2}{2}\right)\theta \right), \ \ \ d_2(\theta, x) = d_1 (\theta, x) - \sigma \sqrt{\theta}$$

\subsection*{Parité Call-Put}
Il faut retenir la relation de parité Call-Put :
$$C_t - P_t = S_t -Ke^{-r(T-t)}$$

\subsection*{Quelques relations utiles}
$$S_t = s_0 \exp \left( \left( \mu - \frac{\sigma^2}{2} \right)t + \sigma B_t \right), \ \ S_t = s_0 \exp \left( \left( r - \frac{\sigma^2}{2} \right)t + \sigma W_t \right)$$ 
$$dS_t = S_t (\mu dt + \sigma dB_t), \ \ dS_t = S_t (r dt + \sigma d W_t), \ \ d\tilde{S}_t = \sigma \tilde{S}_t dW_t$$

\subsection*{Options américaines}
Option américaine vanille : $h_t = \psi(S_t)$. La valeur de cette option $u(t,x)$ est donnée par 
$$u(t,x) = \sup_{\tau \in \mathbb{T}_{t,T}} \E^* \left(e^{-r(\tau-t)} \psi \left( x\exp \left( (r-\sigma^2/2)(\tau - t) + \sigma (W_T-W_t) \right) \right) \right)$$

\underline{Fait à savoir.} Le Call américain et le Call européen ont le même prix.

\section{Résultats théoriques}
\subsection*{Processus d'Itô}
Un processus d'Itô est un processus de la forme
$$X_t = X_0 + \int_0^t K_s ds + \int_0^t H_s dW_s$$
avec 
\begin{itemize}
    \item $X_0$ $\mathcal{F}_0$-mesurable
    \item $H$, $K$ adaptés à la filitration de $W$
    \item $\int_0^T \left(|H_s| + K_s^2\right) ds < + \infty \ \ \prob$-ps 
\end{itemize}
Cette décomposition est unique. 

\underline{Rappel.} 1. Si $\int_0^T H_s^2 ds < + \infty$ $\prob$-ps, alors le processus $\left( \int_0^t H_s dW_s\right)_{t \in [0,T]}$ est un processus $\mathcal{F}_t$-adapté et continu (c'est en fait une martingale locale).

2. Si $\E \left( \int_0^T H_s^2 ds \right) < + \infty$, alors 
le processus $\left( \int_0^t H_s dW_s\right)_{t \in [0,T]}$ est une $\mathcal{F}_t$-martingale continue. 


\subsection*{Théorème de Girsanov}
Soit $(\theta_t)_{t \in [0,T]}$ un processus adapté tel que $\int_0^T \theta_t^2 dt < +\infty$ $\prob$-ps et tel que 
$$L_t = \exp \left( - \int_0^t \theta_s dB_s - \frac{1}{2} \int_0^t \theta_s^2 ds \right)$$
est une $\mathcal{F}_t$-martingale. Alors sous $\mathbb{Q}$ de densité $\frac{d\mathbb{Q}}{d \prob} = L_T$, le processus
$$W_t = B_t + \int_0^t \theta_s ds$$
est un mouvement brownien.

Le critère de la section suivante donne une condition suffisante pour que $L_t$ soit une martingale.

\subsection*{Critère de Novikov}
Si $\E \left( \exp{\left( \frac{1}{2}\int_0^T \theta_t^2 dt \right)} \right) < + \infty$, alors $L_t$ est une $\mathcal{F}_t$-martingale.

\subsection*{EDP de Black-Scholes}
$F(t,x)$ : prix de l'option européenne de payoff $h=f(S_T)$ à la maturité $T$. Elle vérifie l'EDP de Black-Scholes
$$\partial_t F(t,x) + \frac{\sigma^2 x^2}{2}\partial_{xx} F(t,x) + rx \partial_x F(t,x) -rF(t,x)=0$$
avec la condition terminale $F(T,x)=f(x)$.

\subsection*{Isométrie d'Itô et inégalité de Doob}
Si $\E \left( \int_0^T H_t^2 dt\right) < + \infty$, alors $\left( \int_0^t H_s dW_s\right)_{t \in [0,T]}$ est une $\mathcal{F}_t$-martingale vérifiant l'isométrie d'Itô
$$\E \left( \left( \int_0^t H_s dW_s \right)^2 \right) = \E \left( \int_0^t H_s^2 ds\right)$$
et l'inégalité de Doob
$$\E \left( \sup_{s \in [0,T]} \left(\int_0^t H_s dW_s \right)^2\right) \leq  4 \E \left(\int_0^T H_s^2 ds \right)$$

\subsection*{Théorème d'Itô}
Soit $Z \in L^2(\prob)$. Si $b$ et $\sigma$ sont des fonctions continues et lipschitz en $x$ uniformément en $t$, alors l'EDS de diffusion
$$dX_t = b(t, X_t) dt + \sigma(t, X_t) dW_t$$
avec la condition initiale $X_0=Z$ admet une unique solution forte. De plus, il existe $C>0$ telle que 
$$\E\left(\sup_{t \in [0,T]} |X_t|^2 \right) \leq C \E \left( Z^2 \right)$$

\subsection*{Modèle avec dividendes}
L'EDS du modèle est toujours la même, c'est-à-dire 
$$dS_t = S_t(\mu dt + \sigma dB_t)$$
Mais l'équation d'autofinancement s'écrit 
$$dV_t = H_t^0 dS_t^0 + H_t dS_t + H_t \delta S_t dt.$$
L'équation d'autofinancement s'écrit ainsi car dans l'intervalle $[t,t+dt]$, on touche $\delta S_t dt$ pour chaque action qu'on possède.

Avec $\tilde{V}_t = e^{-rt} V_t$, on a 
$$d \tilde{V}_t = \sigma H_t \tilde{S}_t dW_t^\delta \ \ \text{avec} \ \ W_t^\delta = B_t + \frac{\mu+\delta-r}{\sigma}t$$

\underline{Prix du call} :
$$F(t,x) = xe^{-\delta(T-t)} \norm (d_1(T-t,x)) - K e^{-r(T-t)}\norm (d_2(T-t,x))$$
où
$$d_1(\theta, x) = \frac{1}{\sigma \sqrt{\theta}} \left( \ln \left( \frac{x}{K} \right) + (r-\delta+\sigma^2/2) \theta \right), \ \ d_2 (\theta, x) = d_1(\theta, x) - \sigma \sqrt{\theta}$$

\underline{Parité Call-Put} :
$$C_t-P_t = S_t e^{-\delta(T-t)} - K e^{-r(T-t)}$$

\underline{Portefeuille de couverture} :
\[
\begin{cases}
H_t = e^{-\delta (T-t)} \norm\bigl(d_1(T-t,x)\bigr) & \text{(Call)} \\
H_t = e^{-\delta (T-t)} \norm\bigl(d_1(T-t,x)\bigr) & \text{(Put)}
\end{cases}
\]

\subsection{Les grecques}

\begin{table}[h!]
\centering
\resizebox{\columnwidth}{!}{%
\begin{tabular}{|c||c|c|c|c|c|c|}
\hline
 & $S$ & $K$ & $\tau = T-t$ & $\sigma$ & $r$ & $\delta$ \\
\hline\hline
Prix Call européen
& $\uparrow$
& $\downarrow$
& ?
& $\uparrow$
& $\uparrow$
& $\downarrow$ \\
\hline
Prix Put européen
& $\downarrow$
& $\uparrow$
& ?
& $\uparrow$
& $\downarrow$
& $\uparrow$ \\
\hline
Prix Call américain
& $\uparrow$
& $\downarrow$
& $\uparrow$
& $\uparrow$
& $\uparrow$
& $\downarrow$ \\
\hline
Prix Put américain
& $\downarrow$
& $\uparrow$
& $\uparrow$
& $\uparrow$
& $\downarrow$
& $\uparrow$ \\
\hline
\end{tabular}
}
\caption{Sensibilité des prix aux paramètres (flèches de monotonie)}
\end{table}

\end{document}

